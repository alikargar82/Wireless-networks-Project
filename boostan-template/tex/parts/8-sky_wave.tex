% =============================================================================
\hypertarget{chap:sky}{}
\putSec{خوان ششم: آینه‌ای در دل آسمان (موج آسمانی)}
\clearpage

% =============================================================================
\frametitle{۸. راهکار دوم: موج آسمانی (\LR{Sky Wave})}

وقتی هدف ما ارسال سیگنال به آن سوی اقیانوس‌هاست و زمین دیگر توان هدایت موج را ندارد، سرمان را بالا می‌گیریم. در این روش، به جای خزیدن روی زمین، از لایه‌های بالایی جو برای «پرش‌های بلند» استفاده می‌کنیم.



\vspace{0.4cm}
\begin{itemize}
    \setlength\itemsep{0.5cm}
    \item \mybold{فراتر از افق:} برخلاف موج سطحی که با مانع برخورد می‌کرد، موج آسمانی با زاویه‌ای مشخص به سمت فضا پرتاب می‌شود تا در نقطه‌ای بسیار دورتر به زمین برگردد.
    \item \mybold{باند فرکانسی:} این روش تخصصِ فرکانس‌های \mybold{HF} (بین $3$ تا $30$ مگاهرتز) است. 
    \item \mybold{چندین پرش (\LR{Multi-hop}):} سیگنال می‌تواند بعد از برخورد به زمین دوباره به آسمان بازتاب شود و این چرخه را تا هزاران کیلومتر ادامه دهد.
\end{itemize}
\clearpage

% =============================================================================
\frametitle{۹. یونوسفر: آینه‌ی مخفی سیگنال‌ها}

چه چیزی در آسمان وجود دارد که سیگنال را به زمین برمی‌گرداند؟ پاسخ در لایه‌ای به نام \mybold{یونوسفر (\LR{Ionosphere})} نهفته است که از ارتفاع ۶۰ تا ۸۰۰ کیلومتری زمین گسترده شده است.

\vspace{0.3cm}
\begin{myblock}{فرآیند یونیزاسیون: ساخت آینه‌ی رادیویی}
    تابش شدید پرتوهای فرابنفش و ایکس خورشید به مولکول‌های نازک جو در ارتفاعات بالا، باعث جدا شدن الکترون‌ها از اتم‌ها می‌شود. 
    \begin{itemize}
        \item این ابرِ الکترونی، محیطی رسانا ایجاد می‌کند که برای برخی فرکانس‌ها، دقیقاً مثل یک \mybold{سطح فلزی صیقلی} عمل می‌کند.
        \item سیگنال با ورود به این لایه، دچار شکست (Refraction) تدریجی شده و در نهایت به سمت زمین خم می‌شود.
    \end{itemize}
\end{myblock}


\clearpage

% =============================================================================
\frametitle{۱۰. معماری آسمان: لایه‌های D، E و F}

یونوسفر یک پارچه‌ی یکدست نیست؛ بلکه از لایه‌های متفاوتی تشکیل شده که هر کدام رفتار سیگنال را تغییر می‌دهند. این لایه‌ها بر اساس غلظت الکترون‌ها نام‌گذاری شده‌اند:


\vspace{0.5cm}
\vspace{0.4cm}

\begin{center}
    \includegraphics[width=0.7\textwidth]{./images/unisfer.png}\par
	{\small \textbf{شکل 6:} ساختار لایه‌های یونوسفر}
\end{center}

\vspace{0.4cm}
\vspace{0.4cm}
\vspace{0.4cm}

\begin{itemize}
    \setlength\itemsep{0.4cm}
    \item \mybold{لایه D (پایین‌ترین لایه):} در روز تشکیل می‌شود و یک «دشمن» برای سیگنال است. این لایه مثل اسفنج انرژی امواج را جذب کرده و آن‌ها را تضعیف می‌کند.
    \item \mybold{لایه E (لایه میانی):} به بازتاب امواج کمک می‌کند اما با غروب خورشید به سرعت ضعیف می‌شود.
    \item \mybold{لایه F (بالاترین و مهم‌ترین لایه):} قهرمان اصلی انتشار آسمانی است. در روز به دو لایه ($F_1$ و $F_2$) تقسیم می‌شود و در شب با هم ترکیب شده و بهترین بازتاب را فراهم می‌کند.
\end{itemize}
\clearpage

% =============================================================================
\frametitle{۱۱. چرا رادیو شب‌ها شفاف‌تر است؟}

حالا می‌توانیم به سوالی که در بخش رادیو AM مطرح کردیم پاسخ دهیم. تغییرات شبانه‌روزی یونوسفر، کیفیت ارتباطات ما را زیر و رو می‌کند:


\vspace{0.3cm}
\begin{enumerate}
    \setlength\itemsep{0.4cm}
    \item \mybold{در طول روز:} تابش خورشید لایه سمی \mybold{D} را می‌سازد. سیگنال قبل از اینکه به لایه بازتاب‌دهنده اصلی (F) برسد، توسط لایه D بلعیده می‌شود. (به همین دلیل برد رادیو AM در روز کم است).
    \item \mybold{در طول شب:} خورشید می‌رود و لایه D به دلیل نبود تابش، ناپدید می‌شود. 
    \item \mybold{نتیجه:} مسیر برای موج باز می‌شود؛ سیگنال مستقیماً به لایه F برخورد کرده و با کمترین تلفات به هزاران کیلومتر دورتر پرتاب می‌شود.
\end{enumerate}
\clearpage





% =============================================================================
\frametitle{۱۲. وقتی آینه می‌شکند: مفهوم فرکانس بحرانی}

آیا یونوسفر هر سیگنالی را با هر فرکانسی بازتاب می‌دهد؟ خیر! این لایه یک «سقف تحمل» دارد. اگر انرژی (فرکانس) موج خیلی زیاد باشد، یونوسفر دیگر نمی‌تواند آن را خم کند.

\vspace{0.3cm}
\begin{myblock}{تعریف فرکانس بحرانی ($f_c$)}
    حداکثر فرکانسی است که اگر سیگنال آن را \mybold{کاملاً عمودی} (با زاویه ۹۰ درجه) به سمت یونوسفر بفرستیم، همچنان قادر به بازتاب آن به زمین باشد.
\end{myblock}

\vspace{0.3cm}
\mybold{نبرد موج و پلاسما:}
\begin{itemize}
    \item \mybold{اگر $f < f_c$:} تراکم الکترون‌های یونوسفر برای خم کردن مسیر موج کافی است و سیگنال به زمین برمی‌گردد (ارتباط آسمانی برقرار است).
    \item \mybold{اگر $f > f_c$:} انرژی موج بر توانایی شکست لایه غلبه می‌کند. سیگنال، یونوسفر را «سوراخ» کرده و به فضای بیرونی فرار می‌کند.
\end{itemize}
\clearpage





% =============================================================================
\frametitle{۱۳. واقعیت مهندسی: زاویه تابش و \LR{MUF}}

در دنیای واقعی، ما به ندرت سیگنال را عمودی به آسمان می‌فرستیم؛ بلکه آن را با زاویه پرتاب می‌کنیم تا مسافت بیشتری را طی کند.

\vspace{0.4cm}
\vspace{0.4cm}
\vspace{0.4cm}


\begin{center}
   \includegraphics[width=0.7\textwidth]{./images/max_usable_f.jpg}\par
	{\small \textbf{شکل 7:} نمایش زاویه تابش و حداکثر فرکانس قابل استفاده (\LR{MUF})}
\end{center}


\clearpage
\frametitle{۱۳. واقعیت مهندسی: زاویه تابش و \LR{MUF}}

\vspace{0.2cm}
\begin{itemize}
    \item \mybold{تأثیر زاویه:} وقتی موج با زاویه مایل به یونوسفر می‌خورد، راحت‌تر بازتاب می‌شود. بنابراین می‌توانیم از فرکانس‌هایی بالاتر از فرکانس بحرانی هم استفاده کنیم.
    \item \mybold{حداکثر فرکانس قابل استفاده (\LR{MUF}):} بالاترین فرکانسی که در یک زاویه خاص، همچنان بازتاب می‌شود. مهندسان مخابرات HF همیشه باید زیر این مرز حرکت کنند.
\end{itemize}

\vspace{0.4cm}
\begin{myblock}{جمع‌بندی استراتژیک و ورود به عصر جدید}
    پس فهمیدیم که برای فرکانس‌های خیلی بالا (مثل UHF و SHF که در موبایل و ماهواره استفاده می‌شوند)، یونوسفر دیگر آینه نیست، بلکه یک پنجره شفاف است. 
    
    \mybold{سوال بعدی:} حالا که سیگنال‌های مدرن ما از جو عبور می‌کنند، چه چالش‌های جدیدی در مسیر مستقیم (\LR{Line of Sight}) و فضای بیرونی منتظر آن‌هاست؟ (موضوع فصل بعد).
\end{myblock}
\clearpage