% =============================================================================
\putSec{فصل چهارم: فیزیکِ تعامل و تغییرات در مسیر}
\clearpage

% =============================================================================
\frametitle{۲۲. واقعیتِ مسیر: موج در میدان نبرد}

در دنیای واقعی، هیچ سیگنالی در خلأ مطلق حرکت نمی‌کند. هر موجی با هر فرکانسی، در مسیر خود با ۵ پدیده فیزیکی اصلی روبروست. 

\vspace{0.4cm}
\begin{myblock}{یک اصل بنیادین}
    پدیده‌های \mybold{بازتاب، شکست، پراش، پراکنش و جذب} در تمام فرکانس‌ها وجود دارند؛ اما بسته به «طول‌موج» و «جنس محیط»، شدت اثرگذاری یکی از آن‌ها به بقیه غلبه می‌کند.
\end{myblock}



\clearpage

% =============================================================================
\frametitle{۲۳. بازتاب (\LR{Reflection}): برخورد با آینه‌های محیطی}

زمانی رخ می‌دهد که موج به سطحی بزرگتر از طول‌موج خود برخورد کند.

\vspace{0.3cm}
\begin{itemize}
    \setlength\itemsep{0.4cm}
    \item \mybold{رفتار هندسی:} زاویه تابش و بازتاب برابر است. سطوح صیقلی مثل شیشه ساختمان‌ها یا سطح آب، بهترین بازتاب‌دهنده‌ها هستند.
    \item \mybold{در عمل:} بازتاب باعث می‌شود سیگنال از مسیرهای مختلف (Multipath) به ما برسد؛ این اتفاق گاهی باعث تقویت و گاهی باعث خنثی شدن سیگنال می‌شود.
\end{itemize}



\clearpage

% =============================================================================
\frametitle{۲۴. شکست (\LR{Refraction}): خم شدن در مرز محیط‌ها}

شکست، تغییر جهت موج هنگام عبور از محیط‌هایی با چگالی متفاوت است. این پدیده، «مغز متفکر» ارتباطات دوربرد است.

\vspace{0.3cm}
\begin{itemize}
    \setlength\itemsep{0.4cm}
    \item \mybold{شکست جوی:} غلظت هوا در تروپوسفر و یونوسفر یکنواخت نیست. این تغییر غلظت باعث می‌شود سرعت موج تغییر کرده و مسیر آن به سمت زمین «خم» شود.
    \item \mybold{نقش استراتژیک:} اگر شکست نبود، امواج آسمانی هرگز از یونوسفر به زمین برنمی‌گشتند و امواج فضایی نمی‌توانستند انحنای زمین را تا حدی پوشش دهند.
\end{itemize}



\clearpage

% =============================================================================
\frametitle{۲۵. پراش (\LR{Diffraction}): هنرِ دور زدن موانع}

پراش به موج اجازه می‌دهد تا لبه‌های تیز موانع را دور بزند و به «سایه‌ی رادیویی» نفوذ کند.

\vspace{0.3cm}
\begin{itemize}
    \item \mybold{وابستگی به فرکانس:} در فرکانس‌های پایین (رادیو \LR{AM})، پراش بسیار قدرتمند است و کوه‌ها را دور می‌زند. 
    \item \mybold{در فرکانس‌های بالا:} با افزایش فرکانس، پدیده پراش ضعیف می‌شود. به همین دلیل در شبکه \LR{5G}، کوچکترین مانع (مثل یک درخت) می‌تواند ارتباط را قطع کند.
\end{itemize}



\clearpage

% =============================================================================
\frametitle{۲۶. پراکنش (\LR{Scattering}): تفرق در سطوح ناهموار}

اگر ابعاد مانع کوچک‌تر یا در حد طول‌موج باشد (مثل برگ درختان یا قطرات باران)، موج در تمام جهات پخش می‌شود.

\vspace{0.3cm}
\begin{itemize}
    \item \mybold{تفاوت با بازتاب:} در اینجا نظمِ زاویه‌ی بازتاب وجود ندارد و انرژی سیگنال در فضای اطراف پراکنده می‌شود.
    \item \mybold{نتیجه:} باعث می‌شود سیگنال ضعیف شود اما در عین حال امکان دریافت سیگنال در نقاطی که دید مستقیم ندارند را فراهم می‌کند.
\end{itemize}



\clearpage

% =============================================================================
\frametitle{۲۷. جذب (\LR{Absorption}): پایان سفر انرژی}

در این پدیده، انرژی الکترومغناطیسی موج توسط ماده بلعیده شده و به گرما تبدیل می‌شود.

\vspace{0.4cm}
\begin{itemize}
    \item \mybold{موانع جاذب:} دیوارهای بتنی خیس، فلزات و حتی مولکول‌های اکسیژن در فرکانس‌های خاص.
    \item \mybold{تضعیف بارانی:} در فرکانس‌های بالای \LR{10 GHz}، قطرات باران مثل اسفنج انرژی سیگنال را جذب می‌کنند.
\end{itemize}
\clearpage

