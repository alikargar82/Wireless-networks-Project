\hypertarget{chap:space}{}
\putSec{خوان هفتم: عبور از جو (موج فضایی)}
\clearpage

% =============================================================================
\frametitle{۱۴. عصر مدرن: وقتی سیگنال «مستقیم» می‌رود}

در فصول قبل دیدیم که امواج یا روی زمین می‌خزیدند یا از آسمان بازتاب می‌شدند. اما با عبور از مرز \mybold{۳۰ تا ۴۰ مگاهرتز}، قواعد بازی عوض می‌شود. سیگنال ما حالا آنقدر پرانرژی است که لایه‌های جو دیگر برایش «آینه» نیستند، بلکه یک «پنجره شفاف» به سوی فضا هستند.

\vspace{0.3cm}
\begin{itemize}
    \setlength\itemsep{0.4cm}
    \item \mybold{خروج از پیله:} این امواج (\LR{VHF} تا \LR{EHF}) یونوسفر را سوراخ کرده و مستقیم به فضا می‌روند.
    \item \mybold{شخصیت نورگونه:} در این مرحله، موج دقیقاً مثل یک «پرتو نور» رفتار می‌کند؛ فقط در خط مستقیم حرکت کرده و با برخورد به هر مانع کدر، متوقف می‌شود.
    \item \mybold{مأموریت:} این زیربنای دنیای مدرن ماست؛ از رادیو \LR{FM} و تلویزیون تا شبکه‌های موبایل و رادارهای پیشرفته.
\end{itemize}
\clearpage

% =============================================================================
\frametitle{۱۵. استراتژی‌های نبرد: \LR{LoS}، \LR{NLoS} و فراتر از آن}

در این دنیای «خط مستقیم»، موقعیت جغرافیایی تعیین‌کننده نوع ارتباط ماست. ما سیگنال را بر اساس موانع مسیر به سه روش مدیریت می‌کنیم:

\vspace{0.3cm}
\begin{itemize}
    \setlength\itemsep{0.5cm}
    \item \mybold{دید مستقیم (\LR{LoS}):} وضعیت ایده‌آل؛ فرستنده و گیرنده همدیگر را می‌بینند. (مثل دکل‌های مایکروویو نقطه به نقطه).
    
    \item \mybold{بدون دید مستقیم (\LR{NLoS}):} سیگنال پشت ساختمان‌ها پنهان شده و فقط با «بازتاب» و «پراش» به گیرنده می‌رسد. (چالشی که هر روز با گوشی‌های موبایلمان داریم).
    
    \item \mybold{فراتر از افق (\LR{BLoS}):} وقتی فاصله آنقدر زیاد است که انحنای زمین مانع می‌شود. در اینجا یا باید از ماهواره کمک بگیریم و یا از تکرارکننده‌ها (\LR{Repeater}).
\end{itemize}
\clearpage

% =============================================================================
\frametitle{۱۶. جو زمین؛ یک عدسیِ نامرئی (\LR{Refraction})}

اگرچه گفتیم این امواج مستقیم می‌روند، اما اتمسفر هنوز یک برگ برنده دارد. چگالی هوا با ارتفاع تغییر می‌کند و این یعنی سیگنال ما در یک «عدسی غول‌پیکر» در حال حرکت است.

\vspace{0.3cm}
\begin{myblock}{پدیده شکست: کمکِ ناخواسته طبیعت}
    تغییر چگالی باعث می‌شود سرعت موج در ارتفاعات بالاتر کمی بیشتر باشد. نتیجه؟ مسیر مستقیم سیگنال کمی به سمت زمین \mybold{خم می‌شود}.
\end{myblock}

\vspace{0.2cm}
\begin{itemize}
    \item \mybold{افق رادیویی:} به دلیل این خمیدگی، افق رادیویی حدود ۱۵ درصد دورتر از افق واقعی است.
    \item \mybold{قانون ۴/۳:} مهندسان برای ساده‌سازی، فرض می‌کنند زمین کمی بزرگتر است تا این «انحنای مهربان» را در محاسبات لحاظ کنند.
\end{itemize}
\clearpage

% =============================================================================
\frametitle{۱۷. منطقه فرنل: فراتر از یک نگاه ساده}

یک تصور اشتباه در مهندسی این است که «اگر آنتن را می‌بینی، پس ارتباط برقرار است». اما سیگنال برای انتقال انرژی، به یک «فضای حیاتی» در اطراف خط مستقیم نیاز دارد.

\vspace{0.3cm}
\begin{itemize}
    \item \mybold{حباب بیضوی:} موج در مسیری شبیه به یک خربزه یا بیضوی بین دو آنتن پخش می‌شود.
    \item \mybold{قانون ۶۰ درصد:} اگر مانعی (حتی لبه یک ساختمان) وارد این حباب شود، سیگنال را تضعیف می‌کند؛ حتی اگر مانع دقیقاً روی خط مستقیم نباشد!
    \item \mybold{نکته طراحی:} دکل‌ها باید آنقدر بلند باشند که این «فضای تنفس» سیگنال کاملاً خالی بماند.
\end{itemize}
\clearpage

% =============================================================================
\frametitle{۱۸. موانع محیطی: وقتی باران دشمن می‌شود}

هرچه فرکانس بالاتر می‌رود (مثل \LR{5G})، سیگنال «نازک‌نارنجی‌تر» می‌شود! حالا حتی قطرات باران هم برای سیگنال ما تبدیل به سد می‌شوند.

\vspace{0.3cm}
\begin{itemize}
    \setlength\itemsep{0.4cm}
    \item \mybold{تضعیف بارانی (\LR{Rain Fade}):} در فرکانس‌های بالای ۱۰ گیگاهرتز، ابعاد قطرات باران با طول‌موج برابری می‌کند و انرژی سیگنال را می‌بلعد.
    \item \mybold{جذب گازی:} مولکول‌های اکسیژن و بخار آب در برخی فرکانس‌ها مثل اسفنج عمل کرده و انرژی را به گرما تبدیل می‌کنند.
    \item \mybold{دنیای شهری:} بازتاب از دیوارها باعث می‌شود نسخه‌های متعددی از یک پیام به گوشی برسد که گاهی همدیگر را خنثی می‌کنند (\LR{Fading}).
\end{itemize}
\clearpage

% =============================================================================
\frametitle{۱۹. ایستگاه آخر: پیوند با ستارگان}

وقتی سیگنال ما با موفقیت از تلاطم تروپوسفر و غلظت یونوسفر عبور کرد، وارد فضای بیکران می‌شود. جایی که ماهواره‌ها به عنوان «آینه‌های هوشمند» عمل می‌کنند.

\vspace{0.3cm}
\begin{itemize}
    \item \mybold{ماهواره‌های \LR{LEO} و \LR{GEO}:} از استارلینک در نزدیکی زمین تا ماهواره‌های تلویزیونی در ارتفاع ۳۶ هزار کیلومتری.
    \item \mybold{چالش فضایی:} در خارج از جو، دیگر خبری از باران و مه نیست، اما با «کمربندهای پرانرژی وان‌آلن» و نویزهای کیهانی روبرو هستیم.
\end{itemize}

\vspace{0.4cm}
\clearpage