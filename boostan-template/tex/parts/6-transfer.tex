\hypertarget{chap:transfer}{}
\putSec{خوان چهارم: معمای انتقال سیگنال}
\clearpage



\frametitle{۱. معمای انتقال: مأموریت سیگنال در دنیای واقعی}

همه چیز با یک نیاز شروع می‌شود: \mybold{انتقال اطلاعات}. در دنیای مخابرات، ما داده‌ها (صدا، تصویر یا متن) را به رشته‌های صفر و یک تبدیل می‌کنیم؛ اما برای جابجایی این بیت‌ها در فضا، نیاز به یک حامل فیزیکی داریم.

\vspace{0.4cm}
\begin{itemize}
    \setlength\itemsep{0.5cm}
    
    \item \mybold{چالش محیطی:} فضای بین فرستنده و گیرنده، یک خلأ ساده و ساکن نیست. این مسیر سرشار از موانع فیزیکی، تلاطم‌های جوی و تداخل‌هایی است که می‌توانند به راحتی یکپارچگی سیگنال ما را به چالش بکشند.

    \item \mybold{انطباق با طبیعت:} هدف اصلی ما این است که سیگنال را بر اساس ویژگی‌های محیط (اتمسفر و زمین) طراحی کنیم تا این لایه‌های طبیعی، به جای مسدود کردن مسیر، به یک «گذرگاه» برای انتقال سیگنال تبدیل شوند.

\end{itemize}
\clearpage


% =============================================================================


\frametitle{۳. شناسنامه فنی: فرکانس در برابر طول‌موج}
\vspace{0.2cm}

قبل از اینکه سیگنال را راهی سفر کنیم، باید «شخصیت» او را تعیین کنیم. این کار با انتخاب فرکانس انجام می‌شود؛ پارامتری که مشخص می‌کند موج ما در برخورد با اتمسفر و موانع، چه واکنشی از خود نشان دهد.

\vspace{0.5cm}
\begin{myblock}{رابطه طلایی: $c = f \times \lambda$}
    در این ترازوی فیزیکی، سرعت نور ($c$) ثابت است؛ بنابراین فرکانس و طول‌موج رابطه‌ای عکس با هم دارند:
\end{myblock}

\clearpage

\frametitle{۳. شناسنامه فنی: فرکانس در برابر طول‌موج}

\vspace{0.3cm}
\begin{itemize}
    \setlength\itemsep{0.4cm}
    \item \mybold{فرکانس ($f$):} تعداد نوسان در هر ثانیه. فرکانس بالاتر یعنی توانایی حمل دیتای بیشتر، اما در عین حال سیگنال در برابر موانع «کم‌طاقت» می‌شود و زودتر انرژی‌اش را از دست می‌دهد.

    \item \mybold{طول‌موج ($\lambda$):} فاصله‌ی فیزیکی بین دو قله‌ی موج. هرچه طول‌موج بزرگ‌تر باشد، سیگنال صبورتر است و قدرت عجیبی در «دور زدن» موانع و انحنای زمین دارد.
\end{itemize}

\vspace{0.3cm}
\mybold{نکته راهبردی:} انتخاب فرکانس، تعیین‌کننده مسیر حرکت سیگنال است؛ این انتخاب مشخص می‌کند که آیا موج باید روی سطح زمین حرکت کند، از لایه‌های آسمان بازتاب شود و یا به صورت دید مستقیم به مقصد برسد.
\clearpage



% =============================================================================

\frametitle{۴. نقشه راه: ما از کدام مسیر می‌رویم؟}

برای اینکه سیگنال را به گیرنده برسانیم، بسته به فاصله و فرکانسی که انتخاب کرده‌ایم، یکی از ۳ راه اصلی زیر را پیش رو داریم. هر کدام از این مسیرها، قوانین و چالش‌های خاص خودشان را دارند:



\vspace{0.4cm}
\begin{enumerate}
    \setlength\itemsep{0.5cm}
    \item {\bfseries مسیر سطحی (\LR{Ground Wave}):} در این روش، موج به جای فرار به فضا، انحنای زمین را «بغل» می‌کند و مسیرهای طولانی را طی می‌کند. (مختص فرکانس‌های پایین).

    \item {\bfseries مسیر آسمانی (\LR{Sky Wave}):} وقتی فاصله خیلی زیاد است، موج را به آسمان می‌فرستیم تا از لایه‌های جوی مثل یک «آینه» استفاده کند و به مقصد برسد.

    \item {\bfseries مسیر مستقیم (\LR{Space Wave}):} این راهکار دنیای مدرن است؛ شلیک مستقیم سیگنال بین دو آنتن که همدیگر را می‌بینند (مثل موبایل و ماهواره).
\end{enumerate}

\vspace{0.3cm}
\clearpage


% =============================================================================
