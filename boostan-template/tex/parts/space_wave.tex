% =============================================================================
\putSec{فصل سوم: عبور از جو}
\clearpage

% =============================================================================
\frametitle{۱۴. عصر مدرن: چرا موج فضایی (\LR{Space Wave})؟}

وقتی از فرکانس‌های بالای $30 \text{ MHz}$ صحبت می‌کنیم، طول‌موج‌ها آنقدر کوتاه می‌شوند که دیگر نمی‌توانند انحنای زمین را دور بزنند یا از یونوسفر بازتاب شوند. در این مرحله، استراتژی ما به «شلیک مستقیم» تغییر می‌کند.



\vspace{0.3cm}
\begin{itemize}
    \setlength\itemsep{0.4cm}
    \item \mybold{سوراخ کردن آسمان:} این امواج انرژی کافی برای عبور از لایه‌های یونیزه جو را دارند و مستقیماً به فضا می‌روند.
    \item \mybold{رفتار نورگونه:} در این باند (\LR{VHF} تا \LR{EHF})، موج دقیقاً مثل پرتو نور عمل می‌کند؛ یعنی در خط مستقیم حرکت کرده و با برخورد به هر مانع کدر، مسدود می‌شود.
    \item \mybold{کاربردها:} از پخش تلویزیونی و رادیو \LR{FM} گرفته تا شبکه‌های موبایل و رادارهای پیشرفته.
\end{itemize}
\clearpage

% =============================================================================
\frametitle{۱۵. چالش دید مستقیم (\LR{Line of Sight})}

در این روش، فرستنده و گیرنده باید «دید مستقیم» داشته باشند. اما دو عامل اصلی این دید را محدود می‌کنند:



\vspace{0.3cm}
\begin{enumerate}
    \setlength\itemsep{0.4cm}
    \item \mybold{انحنای زمین:} حتی در یک دشت کاملاً تخت، گرد بودن زمین باعث می‌شود که پس از طی مسافتی، آنتن گیرنده پشت افق پنهان شود.
    \item \mybold{موانع جغرافیایی:} ساختمان‌ها، کوه‌ها و حتی پوشش گیاهی غلیظ می‌توانند مسیر مستقیم سیگنال را قطع کنند.
\end{enumerate}
\mybold{راهکار مهندسی:} افزایش ارتفاع آنتن‌ها تنها راه برای عقب راندنِ «افق» و افزایش برد در این روش است.
\clearpage

% =============================================================================
\frametitle{۱۶. جو زمین؛ یک عدسیِ نامرئی (\LR{Refraction})}

سیگنال‌های ما در لایه‌ی پایینی جو (تروپوسفر) حرکت می‌کنند. این لایه یک محیط یکنواخت نیست و با افزایش ارتفاع، غلظت و فشار هوا کم می‌شود.



\vspace{0.3cm}
\begin{myblock}{پدیده شکست (\LR{Refraction})}
    تغییر چگالی هوا باعث تغییر «ضریب شکست» می‌شود. این یعنی لایه‌های بالای جو، سرعت موج را کمی بیشتر از لایه‌های پایین تغییر می‌دهند.
    \begin{itemize}
        \item \mybold{نتیجه:} مسیر مستقیم موج، کمی به سمت زمین \mybold{خم می‌شود}. 
        \item \mybold{مزیت:} این خمیدگی به ما اجازه می‌دهد فراتر از افقِ نوری (آنچه چشم می‌بیند) را پوشش دهیم.
    \end{itemize}
\end{myblock}
\clearpage

% =============================================================================
\frametitle{۱۷. مدل ریاضی: قانون $4/3$ و شعاع مؤثر زمین}

مهندسان برای اینکه محاسباتِ پیچیده‌ی شکستِ جو را ساده کنند، از یک ترفند ریاضی استفاده می‌کنند. آن‌ها فرض می‌کنند که مسیر موج «صاف» است، اما در عوض زمین را کمی بزرگتر تصور می‌کنند!

\vspace{0.4cm}
\begin{itemize}
    \item \mybold{ضریب $K$:} در شرایط استاندارد جوی، این ضریب برابر $1.33$ (یا همان $4/3$) است.
    \item \mybold{محاسبه برد افق رادیویی:} با استفاده از این مدل، مسافت تا افق رادیویی ($d$) بر حسب کیلومتر و ارتفاع آنتن ($h$) بر حسب متر چنین تخمین زده می‌شود:
    $$d \approx 4.12 \times \sqrt{h}$$
\end{itemize}
\mybold{نکته:} این یعنی افق رادیویی حدود ۱۵ درصد دورتر از افق واقعی است.
\clearpage

% =============================================================================
\frametitle{۱۸. منطقه فرنل (\LR{Fresnel Zone}): چرا دیدن کافی نیست؟}

یک تصور اشتباه این است که اگر گیرنده را ببینیم، ارتباط کامل است. اما واقعیت این است که موج نیاز به یک «فضای حیاتی» در اطراف خط مستقیم دارد.



\vspace{0.3cm}
\begin{itemize}
    \item \mybold{بیضوی فرنل:} سیگنال رادیویی شبیه یک لوله باریک نیست، بلکه به شکل یک بیضوی پخش می‌شود.
    \item \mybold{قانون ۶۰ درصد:} اگر موانعی وارد این بیضوی شوند، حتی اگر خط مستقیم را قطع نکنند، باعث تداخل فازی و تضعیف شدید سیگنال می‌شوند.
    \item \mybold{اهمیت در طراحی:} دکل‌ها باید آنقدر بلند باشند که حداقل ۶۰٪ از منطقه اول فرنل کاملاً خالی از مانع باشد.
\end{itemize}
\clearpage

% =============================================================================
\frametitle{۱۹. موانع جوی: وقتی باران دشمن می‌شود}

در فرکانس‌های بالای $10 \text{ GHz}$ (مثل ماهواره‌های تلویزیونی و $5\text{G}$)، ابعاد قطرات باران با طول‌موج سیگنال رقابت می‌کنند.

\vspace{0.3cm}
\begin{itemize}
    \setlength\itemsep{0.4cm}
    \item \mybold{تضعیف بارانی (\LR{Rain Fade}):} قطرات باران انرژی موج را جذب کرده یا آن را پراکنده (\LR{Scattering}) می‌کنند.
    \item \mybold{جذب گازی:} مولکول‌های بخار آب و اکسیژن در فرکانس‌های خاصی (مثل $22$ و $60 \text{ GHz}$) انرژی موج را به شدت می‌بلعند.
\end{itemize}
\mybold{چرا $5\text{G}$ برد کمی دارد؟} یکی از دلایل اصلی، همین حساسیت شدید به موانع جوی و جذب توسط اکسیژن هواست.
\clearpage

% =============================================================================
\frametitle{۲۰. محوشدگی (\LR{Fading}) و مسیرهای چندگانه}

در محیط‌های شهری، سیگنال مستقیم تنها چیزی نیست که به گوشی ما می‌رسد. موج به دیوارها برخورد کرده و نسخه‌های متعددی از آن به گیرنده می‌رسد.



\vspace{0.3cm}
\begin{itemize}
    \item \mybold{تداخل مخرب:} اگر نسخه‌ی بازتابی با اختلاف فاز به گیرنده برسد، می‌تواند سیگنال اصلی را خنثی کند.
    \item \mybold{نتیجه:} این پدیده باعث می‌شود با یک حرکت کوچک گوشی، کیفیت سیگنال تغییر کند \LR{(Multipath Fading)}.
\end{itemize}
\clearpage

% =============================================================================
\frametitle{۲۱. مرز نهایی: ارتباطات ماهواره‌ای}

وقتی سیگنال از تمام لایه‌های جو عبور می‌کند، وارد فضای خلأ می‌شود. ماهواره‌ها در این مرحله نقش «تکرارکننده‌های معلق» را بازی می‌کنند.



\vspace{0.3cm}
\begin{itemize}
    \item \mybold{ماهواره‌های \LR{GEO} و \LR{LEO}:} از فاصله‌ی ۵۰۰ کیلومتری (استارلینک) تا ۳۶ هزار کیلومتری، سیگنال مسیر طولانی را طی می‌کند.
    \item \mybold{کمربند وان‌آلن:} این منطقه مملو از ذرات پرانرژی است که می‌تواند باعث ایجاد «نویز ضربه‌ای» یا آسیب به سخت‌افزار ماهواره شود.
\end{itemize}
\clearpage