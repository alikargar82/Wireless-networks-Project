% --- اسلاید ۳: مکانیزم‌های اصلی ---
\putSec{مکانیزم‌های انتشار} 
\frametitle{مکانیزم‌های اصلی انتشار (\LR{Propagation Mechanisms})}

سه پدیده اصلی که بر سیگنال تأثیر می‌گذارند:

\begin{enumerate}
    \item \textbf{بازتاب (\LR{Reflection}):}
    برخورد موج به مانعی بزرگ‌تر از طول موج (مثل ساختمان).
    
    \item \textbf{تفرّق یا پراش (\LR{Diffraction}):}
    خمش موج در لبه‌های تیز موانع که نقاط کور را پوشش می‌دهد.
    
    \item \textbf{پراکندگی (\LR{Scattering}):}
    برخورد موج با اجسام ریز (مثل قطرات باران یا برگ درختان).
\end{enumerate}
\clearpage

% --- اسلاید ۴: تلفات مسیر ---
\frametitle{تلفات مسیر (\LR{Path Loss})}

\begin{itemize}
    \item \textbf{تعریف:} کاهش توان سیگنال با افزایش فاصله از فرستنده.
\end{itemize}

\begin{myblock}{فرمول فضای آزاد (FSPL)}
    $$L_{dB} = 20 \log_{10}(d) + 20 \log_{10}(f) + 32.44$$
    \begin{itemize}
        \item $d$: فاصله (کیلومتر)
        \item $f$: فرکانس (مگاهرتز)
    \end{itemize}
\end{myblock}

\textbf{نتیجه:} با دو برابر شدن فاصله، توان سیگنال ۶ دسی‌بل افت می‌کند.
\clearpage