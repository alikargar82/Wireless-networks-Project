\hypertarget{chap:events}{}
\putSec{خوان سوم: رويدادهاي اصلي انتشار}
\clearpage


% --- اسلاید ۷: رویدادهای اصلی انتشار ---

\frametitle{رویدادهای اصلی}

در مسیر حرکت موج رادیویی، پنج رویداد فیزیکی اصلی رخ می‌دهد. هر یک از این رویدادها به‌طور مستقل بررسی می‌شود.

این رویدادها عبارتند از:
\begin{enumerate}
	\item \textbf{بازتاب} (\lr{Reflection})
	\item \textbf{جذب} (\lr{Absorption})
	\item \textbf{شکست} (\lr{Refraction})
	\item \textbf{پراش} (\lr{Diffraction})
	\item \textbf{پراکنش} (\lr{Scattering})
\end{enumerate}

%% تصویر: نمایش پنج رویداد اصلی انتشار در یک محیط

\clearpage

% --- اسلاید ۸: بازتاب ---

\frametitle{بازتاب (\LR{Reflection})}

وقتی موج الکترومغناطیسی به سطحی برخورد می‌کند، بخشی از انرژی‌اش در جهتی مشخص برمی‌گردد. این پدیده را بازتاب می‌نامیم.

بازتاب زمانی به‌خوبی رخ می‌دهد که ابعاد سطح انعکاس‌دهنده از طول‌موج بسیار بزرگ‌تر باشد. مثال‌هایی مثل دیوارهای ساختمان، سطح زمین، سطح دریا و فلزات، بسیار خوب موج را بازتاب می‌دهند. برای نمونه، موج \lr{Wi-Fi} (با طول‌موج چند سانتی‌متری) هنگام برخورد به دیوار ساختمان بسیار خوب بازتاب می‌شود.

به‌طور فیزیکی، زاویه تابش برابر با زاویه بازتاب است. اگر موج با زاویه \(30^\circ\) برخورد کند، با همان زاویه بازتاب می‌شود. این قانون را قانون انعکاس می‌نامیم.
\begin{center}
	\includegraphics[width=0.6\textwidth]{./images/reflection}
	\par\vspace{0.2cm}
	{\small \textbf{شکل ۲:} نمایش بازتاب در هنگام برخورد به سطح صاف}
\end{center}


%% تصویر: پرتو موج برخورد‌کننده به سطح صیقلی؛ نمایش زاویه تابش و بازتاب با خط‌های عمودی

\clearpage

% --- اسلاید ۹: جذب ---

\frametitle{جذب (\LR{Absorption})}

بخشی دیگری از انرژی موج، به‌جای بازتاب یا عبور، توسط محیط جذب می‌شود. در جذب، انرژی الکترومغناطیسی به انرژی گرمایی تبدیل می‌شود.

مکانیزم فیزیکی جذب به‌گونه‌ای است که وقتی موج الکترومغناطیسی از محیطی عبور می‌کند، الکترون‌های اتم‌ها و مولکول‌های محیط تحت تأثیر میدان الکتریکی موج قرار می‌گیرند. این الکترون‌ها نوسان می‌کنند و این نوسان باعث تولید گرما می‌شود.

برخی مواد بسیار جاذب‌اند. بخار آب بسیار جاذب است و خاصیت جاذبی‌اش به فرکانس بستگی دارد. مواد حاوی کربن (مثل مقاومت‌ها) خواص جاذب‌کننده قوی دارند. ابر و باران نیز ذرات درون خود باعث جذب شدیدی می‌شوند. نتیجه جذب، تضعیف تدریجی دامنه موج است؛ هرچه مسافت در محیط جاذب بیشتر باشد، تضعیف شدیدتر است.

%% تصویر: موج الکترومغناطیسی عبور‌کننده از محیط جاذب؛ نمایش کاهش تدریجی دامنه

\clearpage

% --- اسلاید ۱۰: شکست ---

\frametitle{شکست (\LR{Refraction})}

وقتی موج از یک محیط به محیط دیگری عبور می‌کند (به‌خصوص اگر خواص فیزیکی دو محیط متفاوت باشد)، جهت انتشار موج تغییر می‌کند. این پدیده را شکست می‌نامیم.

رفتار موج در مرز دو محیط توسط قانون اسنل توصیف می‌شود:

\begin{myblock}{قانون اسنل}
	\[
	n_1 \sin\theta_1 = n_2 \sin\theta_2
	\]
	
	که در آن \(n_1\) و \(n_2\) ضرایب شکست محیط‌های اول و دوم، \(\theta_1\) زاویه تابش (نسبت به عمود بر سطح) و \(\theta_2\) زاویه شکست است.
	
	ضریب شکست برابر است با \(n = \dfrac{c}{v}\) که در آن \(c\) سرعت نور در خلأ و \(v\) سرعت انتشار موج در آن محیط است.
\end{myblock}

به‌عنوان مثال، اگر موج از هوا (ضریب شکست \(n \approx 1\)) وارد آب (ضریب شکست \(n \approx 1.33\)) شود، جهتش خم می‌شود. همین اتفاق در جو نیز می‌افتد، اما به‌صورت تدریجی زیرا ضریب شکست جو به‌آرامی با ارتفاع تغییر می‌کند و مسیر موج تدریجاً خم می‌شود.

%% تصویر: پرتو عبور‌کننده از سطح جداکننده دو محیط با نمایش زاویه تابش و شکست

\clearpage

% --- اسلاید ۱۱: پراش ---

\frametitle{پراش (\LR{Diffraction})}

پراش یکی از مهم‌ترین رویدادهای انتشار است. این پدیده‌ای است که باعث می‌شود موج حول لبه‌های مانع عبور کند.

وقتی موج با مانع برخورد می‌کند، در لبه‌های مانع خم می‌شود و به نواحی «پشت مانع» یا نواحی سایه نیز می‌رسد. پراش بیشتر زمانی قابل‌توجه است که ابعاد مانع قابل‌مقایسه با طول‌موج باشد یا حتی کمتر. برای نمونه، موج با طول‌موج یک متری که با لبه تیزی برخورد کند، به‌خوبی پراش می‌کند.

در عمل، یک کاربر \lr{Wi-Fi} پشت دیوار سیگنال دریافت می‌کند زیرا موج از لبه‌های دیوار پراش کرده و به داخل اتاق می‌رسد. همچنین، یک ساختمان بلند بین فرستنده و گیرنده باعث می‌شود سیگنال از بالای ساختمان پراش کند.

%% تصویر: موج برخورد‌کننده به مانع تیز (Knife-edge)؛ نمایش خم شدن موج حول لبه و رسیدن به ناحیه سایه
\begin{center}
	\includegraphics[width=0.8\textwidth]{./images/difraction}
	    \par\vspace{0.2cm}
	{\small \textbf{شکل ۳:} پراش امواج در اطراف مانع}
\end{center}

\clearpage

% --- اسلاید ۱۲: پراکندگی ---

\frametitle{پراکنش (\LR{Scattering})}

پراکنش رویدادی است که زمانی رخ می‌دهد موج الکترومغناطیسی با ذرات ریزی برخورد کند.

وقتی موج با ذراتی برخورد می‌کند که اندازه‌شان از طول‌موج کمتر یا قابل‌مقایسه باشد، موج در جهت‌های مختلف پخش می‌شود. این پخش‌شدگی را پراکنش می‌نامیم.

منابع پراکنش متنوع‌اند. قطرات باران هر کدام باعث می‌شوند موج کمی تضعیف شود و در جهت‌های مختلف پخش شود. ذرات برف و کریستال‌های یخ نیز پراکنش ایجاد می‌کنند. قطرات آب معلق در مه یا ابر، برگ‌ها و شاخه‌های درختان در محیط‌های جنگلی، و ذرات معلق در هوا مثل گرد و غبار همگی باعث پراکنش می‌شوند. شدت پراکنش به‌شدت به اندازه ذرات و فرکانس موج بستگی دارد. در فرکانس‌های بسیار بالا (مثل باند میلی‌متری)، پراکنش بسیار شدید است.

%% تصویر: موج برخورد‌کننده به ذرات ریز (قطرات باران یا برف)؛ نمایش پخش‌شدگی موج در جهت‌های مختلف

\clearpage

% --- اسلاید ۱۳: خلاصه رویدادها ---

\frametitle{خلاصه و روابط متقابل رویدادها}

\begin{minipage}{0.45\textwidth} % نیمه سمت راست برای متن
	\begin{itemize}
		\item در محیط‌های واقعی، این پنج رویداد معمولاً همزمان رخ می‌دهند. 
		\item یک موج رادیویی در محیط شهری یا جنگلی با تمام این رویدادها رو‌به‌رو است.
		\item بازتاب، جذب، شکست، پراش و پراکندگی همگی بر مسیر و شکل سیگنال اثر می‌گذارند.
		\item درک این رویدادها بنیاد لازم برای طراحی سامانه‌های مخابراتی است.
	\end{itemize}
\end{minipage}
\hfill % ایجاد فاصله بین دو بخش
\begin{minipage}{0.5\textwidth} % نیمه سمت چپ برای تصویر
	\begin{center}
		\includegraphics[width=\textwidth]{./images/p7}
		\par\vspace{0.2cm}
		{\small \textbf{شکل ۴:} نمایش رویدادها در یک قاب}
	\end{center}
\end{minipage} 

\clearpage

