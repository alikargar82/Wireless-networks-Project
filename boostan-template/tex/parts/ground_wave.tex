% =============================================================================
\frametitle{۵. مکانیزم حرکت: فیزیکِ همراهی با زمین}

در فرکانس‌های پایین، سیگنال به جای فرار به سمت فضا، زمین را به عنوان مسیر اصلی خود برمی‌گزیند. عامل اصلی این پدیده، ابعادِ عظیم طول‌موج در این فرکانس‌هاست.
\vspace{0.5cm}

\begin{center}
    \includegraphics[width=0.7\textwidth]{./images/ground_wave_propagation.png}
\end{center}

\clearpage

% =============================================================================

\frametitle{۵. مکانیزم حرکت: فیزیکِ همراهی با زمین}

\vspace{0.3cm}
\begin{itemize}
    \setlength\itemsep{0.5cm}
    \item {\bfseries پدیده پراش (\LR{Diffraction}):} زمانی که طول‌موج ($30 \text{kHz}$ تا $3 \text{MHz}$) در مقایسه با ابعادِ موانع (تپه‌ها یا انحنای افق) بسیار بزرگ باشد، موج آن‌ها را «نمی‌بیند». در نتیجه، سیگنال به جای مسدود شدن، از لبه‌های مانع خم شده و نواحی پشت آن را هم پوشش می‌دهد.
\end{itemize}

\vspace{0.2cm}
\begin{myblock}{نکته فنی: چرا قطبش عمودی؟}
    برای اینکه زمین بتواند موج را هدایت کند، میدان الکتریکی موج حتماً باید {\bfseries عمود بر سطح زمین} باشد. 
    \begin{itemize}
        \item این میدان، جریانی را در سطح زمین القا می‌کند که مثل یک «لنگر»، جبهه‌ی موج را به سمت پایین می‌کشد.
        \item اگر قطبش افقی باشد، زمین مثل یک اتصال‌ کوتاه عمل کرده و بلافاصله انرژی موج را خنثی می‌کند.
    \end{itemize}
\end{myblock}

\vspace{0.2cm}
در واقع زمین در این حالت نقش یک «سیم» یا «موج‌بر» را بازی می‌کند که انرژی را در امتداد انحنای خود هدایت می‌نماید.
\clearpage

% =============================================================================
\frametitle{۶. چالش‌های مسیر: وقتی زمین انرژی را می‌بلعد}

استفاده از زمین به عنوان سیم، رایگان نیست. زمین یک رسانای کامل نیست و در طول مسیر، بخشی از توان سیگنال را به گرما تبدیل می‌کند.



\vspace{0.3cm}
\begin{itemize}
    \setlength\itemsep{0.4cm}
    \item {\bfseries تضعیف و رسانایی خاک:} هرچه زمین رساناتر باشد، موج راحت‌تر حرکت می‌کند. به همین دلیل، برد موج سطحی روی {\bfseries آب دریا} بسیار بیشتر از خاک خشک یا بیابان است.
    
    \item {\bfseries سقف فرکانسی:} با بالا رفتن فرکانس، زمین تشنه‌تر می‌شود و انرژی را سریع‌تر جذب می‌کند. به همین دلیل این روش برای فرکانس‌های بالای $3 \text{MHz}$ عملاً غیرممکن است.
    
    \item {\bfseries کاربرد در دنیای واقعی:} این روش ستون فقرات رادیوهای \LR{AM} و سیستم‌های ناوبری دریایی است؛ چرا که می‌تواند سیگنال را حتی به زیردریایی‌ها در اعماق کم برساند.
\end{itemize}

\vspace{0.2cm}
\mybold{سوال:} اگر بخواهیم فرکانس را بالا ببریم یا از اقیانوس‌ها رد شویم و زمین دیگر یاری نکند، چه باید کرد؟ (پاسخ در راهکار بعدی: موج آسمانی).
\clearpage
% =============================================================================
\frametitle{۷. پخش رادیو \LR{AM}}

رادیو \LR{AM}، کلاسیک‌ترین و ملموس‌ترین نمونه‌ی استفاده از انتشار موج سطحی است که دهه‌هاست برای پوشش‌های محلی و منطقه‌ای به کار می‌رود.


\vspace{0.2cm}
\begin{myblock}{چرا باند \LR{MF} برای رادیو AM انتخاب شد؟}
    این رادیو در فرکانس‌های حدود $530 \text{kHz}$ تا $1700 \text{kHz}$ کار می‌کند.
    \begin{itemize}
        \item \mybold{طول‌موج‌های غول‌پیکر:} در این فرکانس‌ها، طول‌موج بین $175$ تا $560$ متر است! این ابعاد عظیم باعث می‌شود موج به راحتی ساختمان‌ها و تپه‌های شهری را دور بزند (پراش قوی).
    \end{itemize}
\end{myblock}

\vspace{0.2cm}
\begin{itemize}
    \item در طول روز، رادیو AM \mybold{فقط} متکی به موج سطحی است. (در بخش بعدی می‌بینیم که چرا در روز، مسیر آسمان بسته است!).
    \item \mybold{مزیت:} پوشش پایدار و مطمئن محلی بدون نیاز به دید مستقیم.
\end{itemize}
\clearpage