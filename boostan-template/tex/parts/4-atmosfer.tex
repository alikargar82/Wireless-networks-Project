\hypertarget{chap:intro}{}
\putSec{خوان دوم: بسترشناسی}
\clearpage

% --- اسلاید ۲: مروری کلی بر ساختار جو ---

\frametitle{ساختار جو}
\label{chap:intro}
جو زمین محیطی است که موج رادیویی از آن عبور می‌کند. این محیط از چندین لایه متمایز تشکیل شده، هر کدام با خواص فیزیکی و شیمیایی متفاوت. شناخت این لایه‌ها کمک می‌کند بفهمیم سیگنال در فرکانس‌های مختلف، در چه قسمت‌هایی از جو با مسائل روبه‌رو می‌شود.

\begin{itemize}
	\item هر لایه جو دارای دمای خاص، دانسیته متفاوت و ترکیب شیمیایی منحصربه‌فرد است.
	\item برای مخابرات بی‌سیم، دو لایه بسیار مهم‌اند: \textbf{تروپوسفر} (نزدیک زمین) و \textbf{یونوسفر} (ارتفاعات بالا).
	\item سایر لایه‌ها (استراتوسفر، مزوسفر و ترموسفر) نقش‌های کمک‌کننده دارند.
\end{itemize}

\begin{center}
	\includegraphics[width=0.9\textwidth]{./images/atmosphere}
	    \par\vspace{0.2cm}
	{\small \textbf{شکل ۱:} لایه های اتمسفر}


\end{center}


\clearpage

% --- اسلاید ۳: تروپوسفر ---

\frametitle{تروپوسفر: لایه‌ای که در آن زندگی می‌کنیم}

\textbf{تروپوسفر} نزدیک‌ترین لایه به سطح زمین است و از سطح تا حدود \lr{10-18} کیلومتر کشیده شده است. این لایه نقش بسیار مهمی در انتشار امواج رادیویی دارد.

ویژگی‌های فیزیکی تروپوسفر عبارتند از: دمای آن در سطح زمین حدود \lr{15} درجه سلسیوس است و با افزایش ارتفاع، به میزان تقریبی \lr{6} درجه سلسیوس برای هر کیلومتر کاهش می‌یابد. این لایه حاوی بخار آب، ابر، باران، مه و ذرات معلق دیگری است که باعث جذب و پراکندگی امواج رادیویی می‌شوند.

تأثیر تروپوسفر بر فرکانس‌های مختلف متفاوت است. برای فرکانس‌های بسیار پایین، تأثیر آن اندک است؛ اما در فرکانس‌های مایکروویو و بالاتر (\lr{5 GHz} به بالا)، به‌خصوص در شرایط بارانی، جذب و پراکندگی بسیار قابل‌ملاحظه‌اند.

%% تصویر: نمایش لایه تروپوسفر با ابرها، باران و ذرات معلق؛ نمایش تغییر دما با ارتفاع

\clearpage

% --- اسلاید ۴: استراتوسفر و مزوسفر ---

\frametitle{استراتوسفر و مزوسفر: لایه‌های واسط}

\textbf{استراتوسفر} از حدود \lr{18} تا \lr{50} کیلومتر ارتفاع کشیده شده است. برخلاف تروپوسفر، دمای استراتوسفر با ارتفاع تقریباً ثابت است یا حتی افزایش می‌یابد (زیرا ازن نور ماوراء‌بنفش خورشید را جذب می‌کند). نقش مستقیم آن در ارتباطات زمینی عادی کم است، اما بخشی از مسیر امواج الکترومغناطیسی عبوری است.

\textbf{مزوسفر} بین \lr{50} تا \lr{85} کیلومتر قرار دارد و سردترین بخش جو است (دما تا \(-100\) درجه سلسیوس می‌رسد). نقش آن در ارتباطات عادی بی‌سیم اندک است، اما در برخی سامانه‌های رادار بسیار بالا وجود دارد. هر دو این لایه بیشتر به‌عنوان محیط واسط بین تروپوسفر و ترموسفر فعالیت می‌کند.

%% تصویر: مقطع استراتوسفر و مزوسفر؛ نمایش تغییر دما و چگالی هوا

\clearpage

% --- اسلاید ۵: ترموسفر و اگزوسفر ---

\frametitle{ترموسفر و اگزوسفر: بخش‌های بالایی جو}

\textbf{ترموسفر} بین \lr{85} تا \lr{600} کیلومتری ارتفاع قرار دارد. این لایه بسیار گرم است (دما از \lr{500} تا \lr{2000} درجه سلسیوس می‌تواند تجاوز کند). ترموسفر حاوی ذرات باردار (یون‌ها و الکترون‌ها) است و \textbf{یونوسفر} بخشی از ترموسفر است که برای ارتباطات رادیویی فوق‌العاده مهم است.

\textbf{اگزوسفر} بالاتر از \lr{600} کیلومتر و تا حدود \lr{1000} کیلومتری قرار دارد. در این لایه، ذرات به‌قدری پراکنده هستند که رفتار گازی ندارند. محل قرارگیری بسیاری از ماهواره‌های مدار پایین (\lr{LEO}) و معمول (\lr{MEO}) است.

\textbf{کمربندهای ون‌آلن} نواحی حاوی ذرات پرانرژی (الکترون‌ها و پروتون‌ها) در میدان مغناطیسی زمین‌اند. این کمربندها برای طراحی سامانه‌های ماهواره‌ای و فضایی بسیار اهمیت دارند.

%% تصویر: نمایش ترموسفر، اگزوسفر، یونوسفر و کمربندهای ون‌آلن؛ نمایش مدار ماهواره‌ها

\clearpage

% --- اسلاید ۶: یونوسفر و ارتباطات برد بلند ---

\frametitle{یونوسفر: آینه رادیویی برای امواج برد بلند}

\textbf{یونوسفر} بخشی از ترموسفر است (تقریباً \lr{60} تا \lr{1000} کیلومتر ارتفاع) و یکی از مهم‌ترین اجزای جو برای مخابرات بی‌سیم است. این لایه توسط تابش الکترومغناطیسی خورشید یونیزه می‌شود، یعنی اتم‌ها و مولکول‌های هوا الکترون خود را از دست داده و به یون تبدیل می‌شوند.

یونوسفر پر از یون‌ها و الکترون‌های آزاد است. این ذرات باردار می‌توانند با امواج الکترومغناطیسی (به‌خصوص در باندهای فرکانسی پایین مثل \lr{HF}) برهم‌کنش داشته باشند. امواج رادیویی با فرکانس‌های پایین (\lr{3 - 30 MHZ}  در باند \lr{HF}) می‌توانند بازتاب شوند یا تحت تأثیر قرار گرفته و مسیرشان خم شود. این خاصیت باعث می‌شود سیگنال از زمین به یونوسفر برود، آنجا بازتاب شود و دوباره به زمین برگردد.

سیگنال می‌تواند چندین بار بین زمین و یونوسفر رفت و آمد کند (چندین \lr{Hop}) و فاصله‌های هزاران کیلومتری را پوشش دهد. این همان چیزی است که رادیوهای \lr{HF} را قادر می‌سازد ارتباطات برد بسیار بلند برقرار کنند.

%% تصویر: نمایش چندین Hop بین فرستنده و گیرنده با واسطه‌ی یونوسفر

\clearpage
